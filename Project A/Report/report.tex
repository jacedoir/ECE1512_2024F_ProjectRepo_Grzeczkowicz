\documentclass[conference]{IEEEtran}
\IEEEoverridecommandlockouts
% The preceding line is only needed to identify funding in the first footnote. If that is unneeded, please comment it out.
\usepackage{cite}
\usepackage{amsmath,amssymb,amsfonts}
\usepackage{algorithmic}
\usepackage{graphicx}
\usepackage{textcomp}
\usepackage{xcolor}
\usepackage{amsmath,amsfonts,amsthm,amssymb}
\usepackage{setspace}
\usepackage{Tabbing}
\usepackage{enumitem}
\usepackage{fancyhdr}
\usepackage{lastpage}
\usepackage{extramarks}
\usepackage[hidelinks]{hyperref}
\usepackage{chngpage}
\usepackage{caption}
\usepackage{subcaption}
\usepackage{soul,color}
\usepackage{float}
\usepackage{graphicx,float,wrapfig}
% code listing settings
\usepackage{listings}
\lstset{
    language=Python,
    basicstyle=\ttfamily\small,
    aboveskip={1.0\baselineskip},
    belowskip={1.0\baselineskip},
    columns=fixed,
    extendedchars=true,
    breaklines=true,
    tabsize=4,
    prebreak=\raisebox{0ex}[0ex][0ex]{\ensuremath{\hookleftarrow}},
    frame=lines,
    showtabs=false,
    showspaces=false,
    showstringspaces=false,
    keywordstyle=\color[rgb]{0.627,0.126,0.941},
    commentstyle=\color[rgb]{0.133,0.545,0.133},
    stringstyle=\color[rgb]{01,0,0},
    numbers=left,
    numberstyle=\small,
    stepnumber=1,
    numbersep=10pt,
    captionpos=t,
    escapeinside={\%*}{*)}
}

\def\BibTeX{{\rm B\kern-.05em{\sc i\kern-.025em b}\kern-.08em
    T\kern-.1667em\lower.7ex\hbox{E}\kern-.125emX}}
\begin{document}

\title{ECE1512 Project A Report}

\author{\IEEEauthorblockN{Rémi Grzeczkowicz}
\IEEEauthorblockA{\textit{MScAC Student} \\
\textit{University of Toronto}\\
remigrz@cs.toronto.edu}
}

\maketitle

\begin{abstract}
This document is a model and instructions for \LaTeX.
This and the IEEEtran.cls file define the components of your paper [title, text, heads, etc.]. *CRITICAL: Do Not Use Symbols, Special Characters, Footnotes, 
or Math in Paper Title or Abstract.
\end{abstract}

\begin{IEEEkeywords}
component, formatting, style, styling, insert
\end{IEEEkeywords}

\section{Task 1}
\subsection{Part 1}
This question relies on the paper \cite{sajedi2023datadam}.
\begin{enumerate}[label=(\alph*)]
    \item In this paper, the purpose of using Dataset distillation is to reduce the training cost while preserving the performance of the model.
    \vspace{3mm}
    \item The advantages of their methodology over state-of-the-art are:
    \begin{itemize}
        \item It achieved unbiased representation of the real data distribution.
        \item It does not rely on rely on pre-trained network parameters or employ bi-level optimization.
        \item It has a reduced memory cost with a lower run time thanks to the fact that DataDAM does not use an inner-loop bi-level optimization.
        \item It outperformed other distillation methods except for one case where Matching Training Trajectory (MTT) performed better on CIFAR-10 with 10 Impage Per Class (IPC). MIT got an accuracy of $56.5\% \pm 0.7$ while DataDAM got $54.2\% \pm 0.8$.
    \end{itemize}
    \vspace{3mm}
    \item The novelty provided by this paper is the use of attention in data distillation. Indeed it has been used in knowledge distillation but never in dataset distillation.
    \vspace{3mm}
    \item The methodology is as follows:
    \begin{enumerate}[label=(\arabic*)]
        \item Initialize a synthetic dataset $\mathcal{S}$ either using random noise or sampling from the original training dataset $\mathcal{T}$.
        \item For each class $k$ a batch $B_T^k$ of real images and a batch $B_S^k$ of synthetic images are sampled from $\mathcal{T}$ and $\mathcal{S}$ respectively.
        \item Then a neural network $\phi_\theta$ is employed to extract features from the images. The network have different layers, each creating a feature map. This multiple feature maps allow to capture low-level, mid-level and high-level representations of the data.
        \item Using the feature maps of each layer, the Spatial Attention Matching (SAM) module generates an attention map for real and synthetic images. The attention map is formulated as $A(f_{\theta,l}^{T_k}) = \sum_{i=1}^{C_l} | (f_{\theta,l}^{T_k})_i|^p$ where $(f_{\theta,l}^{T_k})_i$ is the $i$th feature map in the $l$th layer, $C_l$ is the number of channels and $p$ is a parameter to adjust the weights of the feature maps.
        \item The attention maps for both datasets are then compared using the loss function $\mathcal{L}_{SAM}$.
        \item The output of the network for each dataset is also compared using the loss function $\mathcal{L}_{MMD}$ based on the Maximum Mean Discrepancy (MMD).
        \item The total loss is then given by $\mathcal{L} = \mathcal{L}_{SAM} + \mathcal{L}_{MMD}$.
        \item Then $\mathcal{S}$ is updated such as $\mathcal{S} = arg \min_{\mathcal{S}} \mathcal{L}$.
    \end{enumerate}
    \vspace{3mm}
    \item DataDAM could be used in machine learning for continual learning by providing an efficient memory management method by storing the synthetic data in the memory instead of the real data. This allows for a better memory usage and a lower computational cost. DataDAM could also be used for neural architecture search. Indeed, instead of training many architectures on the full dataset, those architectures could trained on the distilled dataset, leading to a faster search.
\end{enumerate}

\subsection{Part 2}

\bibliographystyle{plain}
\bibliography{mybibliography}


\end{document}
